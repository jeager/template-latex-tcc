\chapter[Introdução]{Introdução}


\section{Motivação}
Na sociedade atual, cada vez mais milhares de dados são gerados diariamente e há uma dificuldade em se aproveitar ao máximo do conteúdo desses dados, uma vez que é impossível analisar esses dados individualmente para poder identificar o seu real valor. Segundo o Business Intelligence \citeonline{bio2013}, cerca de 90\% dos dados de hoje foram gerados nos últimos
dois anos, tendo como fontes documentos de textos, e-mails, streaming de vídeos e áudio, transações comerciais, telecomunicações e diversas outras. Estimasse que valor investido para trabalhar esses dados tende a superar a faixa de 27 bilhões em 2015 \citeonline{bio2013}.
A fim de analisar esses dados, técnicas foram desenvolvidas e ferramentas foram criadas, tais meios visam facilitar a análise dessa grande massa de dados extraindo valor dela. Um questionário levantado pela Bloomberg \citeonline{bloomberg2011} mostra que 97\% das companhias com rendas acima de 100 milhões de dólares usam algum tipo de análise de negócio, e que essa análise se baseia em grandes volumes de dados e seu valor agregado à companhia.

Para realizar essas análises várias empresas tem feito investimentos na área de Aprendizado de Máquina. A área de Aprendizado de Máquina é considerada um ramo da Inteligência Artificial, sendo umaárea especializada no estudo e construção de sistemas que sejam capazes de aprender de forma automatizada a partir de dados \cite{brink2014}. Dentro da área de Aprendizado de Máquina existem vários passos para se construir um modelo, e uma das áreas existentes é a área de Seleção de Características para ser utilizada no modelo. Uma característica é uma informação que é potencialmente útil para realizar predições \cite{mitchell_1997}. Problemas de Aprendizado de Máquina costumam gerar muitos dados que aparentam ser úteis, mas que nem sempre definem  com exatidão o problema, ou que ajudem a alcançar o objetivo traçado na construção do modelo, de acordo com \citeonline{guyon_2003}, muitos problemas de Aprendizado de Máquina chegam a variar entre cinco mil a cinquenta mil características que, inicialmente, são consideradas importantes ou relevantes ao problema.

Esse grande número de características que são levantadas inicialmente demandam muito tempo para serem processadas, levando a necessidade de reduzir esse número para auxiliar na sua análise. O elevado número ainda acarreta no possível aparecimento de ruídos durante a análise dos dados. A solução para esses problemas é a Seleção de Características. Com essa abordagem é possível escolher os melhores subconjuntos de características que conseguem alcançar os objetivos propostos pelo modelo \cite{molina_2002}. 


\section{Objetivos}
Esse trabalho tem como objetivo principal a implementação de um serviço que seja capaz de analisar um conjunto de características em meio a três modelos de seleção de características e apontar as vantagens e desvantagens de usar cada modelo para seleção.

Em relação aos objetivos específicos do trabalho, a análise das formas de seleção de características assim como os modelos é de fundamental importância para o trabalho, além da implementação de cada um dos modelos.

\section{Estrutura do Trabalho}


