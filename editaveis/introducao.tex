\chapter[Introdução]{Introdução}


\section{Motivação}
Na sociedade atual, cada vez mais milhares de dados são gerados diariamente e há uma dificuldade em se aproveitar ao máximo o conteúdo desses dados, uma vez que é impossível analisar-los individualmente para poder identificar o seu real valor. Segundo o Business Intelligence \citeonline{bio2013}, cerca de 90\% dos dados de hoje foram gerados nos últimos
dois anos, tendo como fontes documentos de textos, e-mails, streaming de vídeos e áudio, transações comerciais, telecomunicações e diversas outras. Estimas-se que valor investido para trabalhar esses dados tende a superar a faixa de 27 bilhões em 2015.

A fim de analisar esses dados, técnicas foram desenvolvidas e ferramentas foram criadas, tais meios visam facilitar a análise dessa grande massa de dados e extraindo valor dela. Um questionário levantado pela Bloomberg \citeonline{bloomberg2011} mostra que 97\% das companhias com receitas acima de 100 milhões de dólares usam algum tipo de análise de negócio, e que essa análise se baseia em grandes volumes de dados e seu valor agregado à companhia.

Para realizar essas análises, várias empresas tem feito investimentos na área de Aprendizado de Máquina, em inglês \textit{Machine Learning (ML)}. A área de Aprendizado de Máquina é considerada um ramo da Inteligência Artificial, sendo uma área especializada no estudo e construção de sistemas que sejam capazes de aprender de forma automatizada a partir de dados \cite{brink2014}. Dentro da área de Aprendizado de Máquina existem vários passos para se construir um modelo, e um desses passos é o Processo de Seleção de Características em que um dado potencialmente útil para realizar predições \cite{mitchell_1997}. 

Problemas de Aprendizado de Máquina costumam gerar muitos dados que aparentam ser úteis, mas que nem sempre definem  com exatidão o problema, ou que ajudem a alcançar o objetivo traçado na construção do modelo. De acordo com \citeonline{guyon_2003}, muitos problemas de Aprendizado de Máquina chegam a variar entre cinco mil a cinquenta mil características que, inicialmente, são consideradas importantes ou relevantes ao problema. Esse grande número de características que são levantadas inicialmente demandam muito tempo para serem processadas, levando a necessidade de reduzir esse número para auxiliar na sua análise. O elevado número ainda acarreta no possível aparecimento de ruídos durante a análise dos dados. A Seleção de Características visa auxiliar esse problema motivado pelas técnicas para lidar com esses dados. Com essa abordagem é possível escolher os melhores subconjuntos de características que conseguem alcançar os objetivos propostos pelo modelo. 

\section{Objetivos}

Esse trabalho teve como objetivo principal a implementação de um serviço que fosse capaz de analisar um conjunto de características e extrair o melhor subconjunto possível analisando-o em meio a três modelos de seleção de características.

\section{Objetivos Específicos}

Para poder alcançar o objetivo geral, esse trabalho foi divido em etapas. Os objetivos específicos do trabalho são:

\begin{enumerate}
	\item{Especificar o processo de seleção de características}
	\item{Pesquisar modelos de seleção de características a serem utilizados}
	\item{Implementar os modelos de seleção de características}
	\item{Realizar testes nos modelos}
	\item{Específicar arquitetura do serviço}
	\item{Construir protótipo do serviço}
	\item{Realizar testes no protótipo do serviço}
\end{enumerate}


\section{Justificativa}

Como já foi pontuado, a sociedade necessita de métodos e ferramentas para conseguir processar a avalanche de informação produzida por segundo. Nesse sentido, o tema deste trabalho se mostra muito pertinente, uma vez que o processo de seleção de características pode viabilizar processamento de massas de dados extremamente extensas, evidenciando as variáveis mais relevantes para o problema alvo do estudo, reduzindo assim o esforço computacional. 

Tão importante como a concepção de novos modelos ou a criação de ferramentas para o processamento de dados é a simplificação do uso dessas tecnologias por profissionais, cuja densidade em programação, integração, modelos estatísticos/matemáticos seja limitada. Essa facilidade é o que motiva a proposta de desenvolvimento de um serviço para o cientista de dados poder realizar o processo de seleção de características, sem muita dificuldade

\section{Método de Pesquisa}

\subsection{Etapa 1: Especificar o processo de seleção de características}

A primeira etapa deste trabalho consistiu em elucidar e específicar como funciona o processo para seleção de características. Para tal, feito foi necessário fazer uma pesquisa bibliográfica visando demonstrar o fluxo do processo de seleção e suas principais etapas.

\subsection{Etapa 2: Pesquisar modelos de seleção de características a serem utilizados}

Essa etapa consistiu em pesquisar os modelos existentes e como são compostos, levando em consideração o processo estudado na etapa anterior e se esses modelos estão consistentes ao processo de seleção de características em si. Após realizada as pesquisas foram escolhidos os modelos que são utilizados no serviço implementado.

\subsection{Etapa 3: Implementar os modelos de seleção de características}

Para cada um dos modelos selecionados, foi realizado a sua implementação através da biblioteca Weka \cite{weka_2005}. Essa escolha foi feita visando uma melhor compatibilidade entre os modelos.

\subsection{Etapa 4: Realizar testes nos modelos}

Os modelos implementados foram testados utilizando duas base de dados, a Madelon \cite{madelon_2003} e a Titanic. Essas bases de dados foram pré-processadas para preencher as lacunas na base de dados. Os resultados foram obtidos e documentados.

\subsection{Etapa 5: Específicar arquitetura do serviço}

Nessa etapa foi feito a especificação de como o serviço proposto por esse trabalho funcionaria, descrevendo cada uma de suas fases e como se dá o seu fluxo.

\subsection{Etapa 6: Construir protótipo do serviço}

Após feito a especificação da arquitetura, foi feito um protótipo do serviço, com todas as funcionalidades levantadas nas etapas anteriores. Esse protótipo recebe uma base de cados e faz o processo de seleção utilizando o modelo selecionado, extraindo o melhor subconjunto de caraterísticas e mostrando a acurácia alcançada por esse subconjunto.

\subsection{Etapa 7: Realizar testes no protótipo do serviço}
Com a implementação do protótipo do serviço, foram feitos testes para garantir que as funcionalidades implementadas estão funcionando de acordo com o projeto. Foram utilizadas as mesmas bases e os resultados obtidos são iguais aos obtidos ao testar os modelos fora do serviço.
