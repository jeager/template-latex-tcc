\chapter[Desenvolvimento]{Desenvolvimento}

Esse trabalho propõe o desenvolvimento de um serviço que seja capaz de receber uma base de dados, submete-la aos modelos propostos, e então entregar ao usuário uma análise de quais características foram escolhidas por cada um dos modelos e qual o ganho obtido em questão de acurácia de predição.

No capítulo introdutório deste trabalho foram estabelecidas oito etapas a serem realizadas. Logo abaixo temos um resumo do que foi feito em cada uma dessas etapas e qual o respectivo resultado.

\section{Etapa 1: Especificar o processo de seleção de características}

Essa etapa trouxe a fundamentação teórica necessária ao tema explorado por esse trabalho. Tal investigação permitiu esclarecimentos essenciais sobre práticas e modelos que subsidiaram a construção do serviço. Por se tratar de um conteúdo de suma importância para esse trabalho, e um tanto quanto extenso, foi decidido criar um capítulo para que fosse feito todo o levantamento bibliográfico referente ao processo de seleção de características (Capítulo \ref{ch:caracteristicas}). Neste capítulo foi descrito as diversas formas de combinação de técnicas a fim de demonstrar os vários modelos que se é possível construir.

\subsection{Etapa 2: Pesquisar modelos de seleção de características a serem utilizados}

Após realizada a pesquisa para o entendimento do processo de seleção, foi necessário realizar uma pesquisa para escolher quais modelos seriam utilizados, uma vez que existem vários, e que a sua escolha é de fundamental importância na execução desse trabalho. 

Os modelos foram escolhidos após pesquisas e análises em relação a dificuldade de implementação, uma vez que a complexidade pode ser alta, além do tempo gasto para realizar os testes da etapa seguinte, que poderia demandar muito tempo. Os modelos selecionados foram: Relief-F \cite{dash_1997}, DTM \cite{dash_1997}, e LFS \cite{gutlein_2009}. Os modelos estão melhor descritos no Capítulo \ref{ch:modelos}.

\subsection{Etapa 3: Implementar os modelos de seleção de características}

Para realizar a implementação foram realizadas pesquisas de ferramentas de mineração de dados. Durante as pesquisas foram encontradas ferramentas capazes de auxiliar o projeto, sendo as que mais se destacam: Scikit-learn \cite{scikit}, Weka \cite{weka_2005} e Apache Spark \cite{spark}. Dentre as ferramentas, a que mais se destacou a esse trabalho foi a Weka, pelo fato dela ter alguns dos métodos já implementados, e pelo fato dela ser uma biblioteca código aberto, o que facilitaria caso alguma mudança fosse necessária. A biblioteca também possui uma grande popularidade e comunidade, o que facilita na busca por auxilio ao seu uso.

A partir da seleção de uma biblioteca, foi então implementado cada um dos modelos escolhidos para serem utilizados no serviço.

\subsection{Etapa 4: Realizar testes nos modelos}

Conforme será mostrado no Capitulo \ref{ch:validacao}, os modelos foram testados utilizando duas bases de dados: a base dos passageiros do Titanic e a base MADELON, utilizada em competições de aprendizado de máquina. 

\subsection{Etapa 5: Específicar arquitetura do serviço}

Durante essa etapa foi pensado como deveria ser o funcionamento da aplicação, uma vez que a proposta desse trabalho era criar um serviço que pudesse submeter uma base de dados e extrair as melhores características de acordo com o modelo selecionado. Como os modelos já haviam sido implementados utilizando a biblioteca Weka, foi então necessário encontrar um conjunto de ferramentas que se adequasse à biblioteca, e que pudesse desenvolver uma aplicação web. Levando em consideração essas premissas foram escolhidos as ferramentas Ruby on Rails \cite{ror} e a linguagem JRuby \cite{jruby}. A combinação da linguagem JRuby com a ferramenta Ruby on Rails, permitiria utilizar o poder e agilidade de desenvolvimento Web da ferramenta Ruby on Rails, junto ao uso de bibliotecas implementadas em Java através do JRuby, o que se encaixaria perfeitamente no contexto desse trabalho.

Ao tentar realizar a comunicação das bibliotecas, alguns problemas foram encontrados, como por exemplo as divergências entre a estrutura da documentação do Weka com a sua estrutura de classes, bibliotecas Ruby que não eram compatíveis com JRuby e tratamento de exceções da biblioteca Weka. Esse problemas foram resolvidos ao longo do dessa etapa, e consolidados na Etapa 6, onde foi implementado o serviço.

\subsection{Etapa 6: Construir protótipo do serviço}

Após a devida integração entre as ferramentas e os componentes do sistema, foram criadas as entidades da aplicação, e a partir delas começou-se a modelar como seria o serviço. Seguindo o que foi proposto no diagrama da arquitetura e as fases que foram propostas, o sistema começou a tomar forma. Um ponto crucial da construção foi extrair os modelos de seleção da ferramenta Weka, uma vez que acessar suas classes e montar os seus modelos foi trabalhoso, porém fez com que o serviço fizesse exatamente o que foi proposto. Tendo como resultado desta etapa um serviço que receberia uma base de dados, faria seu pre processamento, a executaria, e em seguida forneceria os resultados dessa execução.


\subsection{Etapa 7: Realizar testes no protótipo do serviço}

Para que fosse possível aferir a funcionalidade do serviço, as mesmas bases utilizadas na etapa 4, foram utilizadas para que fosse possível verificar se o desempenho seria mantido. Além desse teste, foram testados também bases de dados com estruturas diferentes da proposta na fase de Pré Processamento (\ref{sec:preprocessamento}) e o sistema deveria recusá-las. Em ambos os casos o resultado foi positivo. O desempenho do sistema foi mantido em relação ao que foi feito anteriormente, e o sistema recusava as bases sem os pré requisitos de estrutura, mais detalhes podem ser observados no Capítulo \ref{ch:servico}