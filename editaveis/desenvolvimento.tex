\chapter[Desenvolvimento]{Desenvolvimento}

Esse trabalho propõe o desenvolvimento de um serviço que seja capaz de receber uma base de dados, submete-la aos modelos propostos, e então entregar ao usuário uma análise de quais características foram escolhidas por cada um dos modelos e qual o ganho obtido em questão de acurácia de predição.

No capítulo introdutório desse trabalho foram estabelecidas oito etapas a serem realizadas. Logo abaixo temos um resumo do que foi feito em cada uma dessas etapas e qual o resultado obtido em cada uma delas.

\section{Etapa 1: Especificar o processo de seleção de características}

Essa etapa serviu de base para esse trabalho, sendo dela a origem da inspiração para elaborar um serviço capaz de analisar as características de um dado problema e selecioná-las. O processo de seleção de características foi descrito a parte no Capítulo \ref{ch:caracteristicas} por ser de fundamental importancia o seu entendimento para esse trabalho. Neste capítulo foi descrito as diversas formas de combinação de técnicas a fim de demonstrar os vários modelos que se é possível construir.

\subsection{Etapa 2: Pesquisar modelos de seleção de características a serem utilizados}

Após realizado a pesquisa para o entendimento do processo de seleção, foi necessário realizar uma pesquisa para escolher quais modelos seriam utilizados, uma vez que existem vários, e que a sua escolha é de fundamental importância na execução desse trabalho. 

Os modelos foram escolhidos após pesquisas e análises de viabilidade. Um dos pontos levados em consideração para a seleção dos modelos foi a sua implementação, uma vez que a complexidade pode ser alta, além do tempo gasto para realizar os testes da etapa seguinte, que poderia demandar muito tempo. Os modelos selecionados foram: Relief-F \cite{dash_1997}, DTM \cite{dash_1997}, e LFS \cite{gutlein_2009}. Os modelos estão melhor descritos no Capítulo \ref{ch:modelos}, onde é possível verificar o estudo que foi feito sobre cada um deles.

\subsection{Etapa 3: Implementar os modelos de seleção de características}

Os modelos escolhidos deveriam ser implementados, para posteriormente serem testados. Para realizar a implementação foram realiadas pesquisas e foi encontrado a ferramenta Weka. A ferramenta Weka é composta por um conjunto de algoritmos de aprendizado de máquina e de mineração de dados. \cite{weka_2005}. Esses algoritmos podem ser utilizados na própria ferramenta como também podem ser utilizados em códigos java, através da biblioteca Weka, que é de código aberto e possui licença GPL 3 \cite{gpl_2007}.

Os algoritmos utilizados nesse trabalho se encontram disponíveis na ferramenta, sendo assim, os modelos foram utilizados a partir da ferramenta, e logo em seguida implementados utilizando a biblioteca fornecida pela mesma para serem utilizados no Serviço que foi implementado.

\subsection{Etapa 4: Realizar laboratório nos modelos}

Conforme mostrado no Capitulo \ref{ch:validacao}, os modelos foram testados utilizando duas bases de dados: a base dos passageiros do Titanic e a base MADELON, utilizada em competições de aprendizado de máquina. 

Os resultados obtidos e toda a metodologia utilizada para alcança-los foram documentadas. Esses resultados foram satisfatórios do ponto de vista experimental do trabalho, uma vez que houve ganhos próximos a 40\% na acurácia do classificador utilizado.

\subsection{Etapa 5: Específicar arquitetura do serviço}

Durante essa etapa foi pensado como deveria ser o funcionamento da aplicação, uma vez que a proposta desse trabalho era criar um serviço que pudesse submeter uma base de dados e extrair as melhores características de acordo com o modelo selecionado. Através de pesquisas, foi proposto um modelo que pode ser melhor visto na Figura \ref{fig:fig13}. Para contemplar essa arquitetura foi necessário buscar ferramentas que fossem capazes de poder integrar a biblioteca Weka, que foi utilizada para executar os modelos, e que auxiliasse a criação de um sistema Web. Levando em consideração essas premissas foram escolhidos as ferramentas Ruby on Rails \cite{ror} e a linguagem JRuby \cite{jruby}. A combinação da linguagem JRuby com a ferramenta Ruby on Rails, permitiria utiliza o poder e agilidade de desenvolvimento Web da ferramenta Ruby on Rails, junto ao uso de bibliotecas implementadas em Java, o que se encaixaria perfeitamente no contexto desse trabalho.

Após várias tentativas foi possível realizar a integração das ferramentas e a partir disso foi dado o inicio da Etapa 6, que consistiria em construir o protótipo do serviço.

\subsection{Etapa 6: Construir protótipo do serviço}

Após ter as ferramentas se comunicando e funcionando, foram criadas as entidades do sistema, e a partir delas começou-se a modelar como seria a aplicação, seguindo o que foi proposto no diagrama da arquitetura (Figura \ref{fig:fig13}). Seguindo-se as fases que foram propostas (\ref{sec:fases}) e o fluxograma de execução do serviço (Figura \ref{fig:fig14}), o sistema começou a tomar forma. Um ponto crucial da construção foi extrair os modelos de seleção da ferramenta Weka, uma vez que acessar suas classes e montar os seus modelos foi trabalhoso, porém fez com que o serviço fizesse exatamente o que foi proposto. Tendo como resultado dessa etapa um serviço que receberia uma base de dados, faria seu pre processamento, executaria, e em seguida forneceria os resultados dessa execução.


\subsection{Etapa 7: Realizar laboratório no protótipo do serviço}

Para que fosse possível aferir a funcionalidade do serviço, as mesmas bases utilizadas no Capítulo \ref{ch:validacao}, foram utilizadas para que fosse possível verificar se o desempenho seria mentido. Além desse teste, foram testados também bases de dados com estruturas diferentes da proposta na fase de Pré Processamento (\ref{sec:preprocessamento}) e o sistema deveria recusá-las. Em ambos os casos o resultado foi positivo. O desempenho do sistema foi mantido em relação ao que foi feito anteriormente, e o sistema aceitava apenas bases que obedecessem ao que foi estabelecido.

\subsection{Etapa 8: Análise dos resultados}