\chapter[Considerações Finais]{Considerações Finais}

Este projeto foi composto por duas etapas, onde a primeira etapa consistiu em realizar um estudo abrangente sobre o Processo de Seleção de Características, os modelos existentes, e a maneira como eles são aplicados, além de uma coleta de resultados de modelos escolhidos para serem implementados. Já na segunda parte foi realizada a implementação de um serviço que capaz de receber uma base, realizar um pré processamento, e então, de acordo com a escolha do usuário, aplicar um dos modelos implementados sobre essa base, e posteriormente informando os resultados obtidos em relação a acurácia inicial e após a seleção de caraterísticas, tempo gasto e as características selecionadas.

\section{Trabalho Realizado}

As etapas que foram propostas nesse trabalho foram descritas no capítulo introdutório, resumidas no capítulo de Desenvolvimento, e agora serão apontadas as principais dificuldades em cada uma delas:

\begin{enumerate}
	\item{Especificar o processo de seleção de características - O levantamento bibliográfico para que essa etapa fosse possível foi exaustivo. Apesar de haver uma documentação razoável, boa parte dela estava incompleta, sendo necessário fazer um quebra cabeças para que essas partes se encaixassem;}
	\item{Pesquisar modelos de seleção de características a serem utilizados - Os modelos encontrados eram muito bem expostos, porém houve uma certa dificuldade em entender o funcionamento de alguns deles, o que tomou um certo tempo;}
	\item{Implementar os modelos de seleção de características - Uma das partes que demandou muito tempo devido a necessidade de implementar modelos diferentes, mas que foi fortemente auxiliado pela ferramenta Weka, escolhida para dar suporte a implementação dos modelos e do classificador utilizado;}
	\item{Realizar testes nos modelos - Aplicar o que ja havia sido feito não foi tão difícil, porém coletar os resultados demandou um pouco de tempo;}
	\item{Específicar arquitetura do serviço - Encontrar um conjunto de ferramentas que fosse capaz de suprir as necessidades e ao mesmo tempo prover agilidade no desenvolvimento de um sistma foi o mais difícil nessa etapa, logo em seguida vem a dificuldade de fazer elas se comunicarem e funcionarem de maneira correta;}
	\item{Construir protótipo do serviço - A parte de desenvolvimento foi árdua. Várias idéias de aprimoramento surgiam ao longo do desenvolvimento, além da vontade de entregar um sistema que estivesse suficientemente maduro. Ao final desse capítulo estão enumeradas algumas das idéias que foram surgindo ao longo do desenvolvimento e que não puderam ser implementadas em tempo hábil;}
	\item{Realizar testes no protótipo do serviço - O empacotamento do que foi feito no sistema e se o que foi feito condizia com os resultados apontados anteriormente saiu de maneira natural. Essa foi uma etapa de colocar e evidenciar as características da ferramenta neste documento.}
\end{enumerate}

Após realizadas todas as etapas foi possível alcançar um serviço que condiz com o objetivo inicial desse trabalho, que era: "Esse trabalho tem como objetivo principal a implementação de um serviço que fosse capaz de analisar um conjunto de características e extrair o melhor subconjunto possível analisando-o em meio a três modelos de seleção de características". Os três modelos foram desenvolvidos e implementados junto ao serviço, e o serviço foi concluído utilizando um conjunto de tecnologias que permitiu o seu desenvolvimento e ainda deixando possíveis espaços para novos modelos. O trunfo desse trabalho é poder dar ao usuário o poder de realizar a seleção de características sem saber muito sobre o processo de seleção em si, e sem saber codificar, apenas necessita da base de dados a ser executada e que ela siga os padrões estabelecidos nos capítulos anteriores.

\section{Trabalhos Futuros}

Apesar do trabalho ter tido bons frutos, ainda existem possiveis melhorias a serem feitas, pois não houve tempo hábil para implementar mais funcionalidades. Abaixo estão enumeradas alguns futuros trabalhos que podem ser realizados.

\begin{enumerate}
	\item{Desevolver mais modelos: Como mostrado nesse trabalho, existem vários modelos que podem ser implementados}
	\item{Pesquisa no desenvolvimento de mais modelos: Utilizando as diversas técnicas de busca, avaliação e seleçào, podem-se criar vários novos modelos.}
	\item{Utilizar outros classificadores: Nesse trabalho foi utilizado o classificador kNN, mas pode-se fazer o estudo utilizando outros classficadores}
	\item{Deixar o serviço mais customizável: Apesar do serviço ser um tanto flexível quanto aos modelos, pode-se colocar mais classficadores, combinações de modelos, etc.}
\end{enumerate}

Tais avanços seriam de grande auxílio para o sistema, deixando-o ainda mais completo e tornando-o uma ferramenta ainda mais poderosa.