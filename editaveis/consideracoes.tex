\chapter[Considerações Finais]{Considerações Finais}

Este projeto é composto por duas etapas, sendo este trabalho a primeira e o Trabalho de Conclusão de Curso 2 a segunda, a ser realizada em 2016. A primeira parte consistiu na fundamentação teórica e na demonstração dos algoritmos a serem utilizados no serviço que será implementado. Consistiu também na experimentação dos algoritmos e verificação da sua eficácia, como apontado no capítulo 4. Já a segunda parte, que consistirá na implementação do serviço de submissão de base de dados e seleção de características. 

\section{Trabalho Realizado}

As etapas propostas nesse trabalho foram descritas no capítulo introdutório, e resumidas no capítulo de Desenvolvimento, sendo elas:

\begin{enumerate}
	\item{Especificar o processo de seleção de características}
	\item{Pesquisar modelos de seleção de características a serem utilizados}
	\item{Implementar os modelos de seleção de características}
	\item{Realizar laboratório nos modelos}
	\item{Específicar arquitetura do serviço}
	\item{Construir protótipo do serviço}
	\item{Realizar laboratório no protótipo do serviço}
	\item{Analiser os resultados}
\end{enumerate}

Após realizadas todas as etapas foi possível alcançar um serviço que condiz com o objetivo inicial desse trabalho, que era: 'Esse trabalho tem como objetivo principal a implementação de um serviço que fosse capaz de analisar um conjunto de características e extrair o melhor subconjunto possível analisando-o em meio a três modelos de seleção de características'. Os três métodos foram desenvolvidos e implementados junto ao serviço, e o serviço foi concluído utilizando um conjunto de tecnologias que permitiu o seu desenvolvimento e ainda deixando possíveis espaços para novos modelos. O trunfo desse trabalho é poder dar ao usuário o poder de realizar a seleção de características sem saber muito sobre o processo de seleção em si, e sem saber codificar, apenas necessita da base de dados a ser executada e que ela siga os padrões estabelecidos nos capítulos anteriores.

Depois de revisitar todo o conteúdo estudado e gerado nesse trabalho, pode-se concluir que foi satisfatório o desempenho alcançado com a geração do serviço e que o trabalho deixa continuidade para futuras melhoras e implementações.

\subsection{Cronograma}

Esse trabalho foi realizado seguindo o cronograma apresentado na tabela abaixo. Cada etapa possui sua duração maracada com um 'X'. Esse cronograma traz consigo apenas os meses válidos nos semestres letivos da Universidade de Brasília, sendo assim, os meses de Janeiro e Fevereiro de 2016 foram removidos do cronograma.

\begin{table}[H]
\centering
\caption{Cronograma}
\label{my-label}
\begin{tabular}{|l|l|l|l|l|l|l|l|l|l|l|}
\hline
\rowcolor[HTML]{C0C0C0} 
Atividade                       & Ago & Set & Out & Nov & Dec & Mar & Abr & Mai & Jun & Jul \\ \hline
\cellcolor[HTML]{C0C0C0}Etapa 1 & X   & X   &     &     &     &     &     &     &     &     \\ \hline
\cellcolor[HTML]{C0C0C0}Etapa 2 &     & X   & X   &     &     &     &     &     &     &     \\ \hline
\cellcolor[HTML]{C0C0C0}Etapa 3 &     &     & X   & X   & X   &     &     &     &     &     \\ \hline
\cellcolor[HTML]{C0C0C0}Etapa 4 &     &     &     &     & X   &     &     &     &     &     \\ \hline
\cellcolor[HTML]{C0C0C0}Etapa 5 &     &     &     &     &     & X   &     &     &     &     \\ \hline
\cellcolor[HTML]{C0C0C0}Etapa 6 &     &     &     &     &     & X   & X   & X   &     &     \\ \hline
\cellcolor[HTML]{C0C0C0}Etapa 7 &     &     &     &     &     &     &     & X   & X   &     \\ \hline
\cellcolor[HTML]{C0C0C0}Etapa 8 &     &     &     &     &     &     &     &     & X   & X   \\ \hline
\end{tabular}
\end{table}


\section{Trabalhos Futuros}

Apesar do trabalho ter tido bons frutos, ainda existem possiveis melhoras a serem feitas, pois não houve tempo hábil para implementar mais funcionalidades. Abaixo estão enumeradas alguns futuros trabalhos que podem ser realizados.

\begin{enumerate}
	\item{Desevolver mais modelos: Como mostrado nesse trabalho, existem vários modelos que podem ser implementados}
	\item{Pesquisa no desenvolvimento de mais modelos: Utilizando as diversas técnicas de busca, avaliação e seleçào, podem-se criar vários novos modelos.}
	\item{Utilizar outros classificadores: Nesse trabalho foi utilizado o classificador kNN, mas pode-se fazer o estudo utilizando outros classficadores}
	\item{Deixar o serviço mais customizável: Apesar do serviço ser um tanto flexível quanto aos modelos, pode-se colocar mais classficadores, combinações de modelos, etc.}
\end{enumerate}

Esseś possíveis trabalhos seriam de grande auxílio para o sistema, deixando-o ainda mais completo e tornando-o uma ferramenta ainda mais poderosa.