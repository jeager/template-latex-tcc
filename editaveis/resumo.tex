\begin{resumo}
A área de Aprendizado de Máquina tem crescido muito e ganhado muitos adpetos, o que faz com que cada vez mais seja necessário aprender a lidar e tratar o gigante volume de dados gerados diariamente por diversos usuários espalhados pelo mundo. Tendo em vista toda essa massa de dados, é preciso saber selecionar quais deles são realmente relevantes. Para poder melhor selecionar esses dados existe o Processo de Seleção de Características. Este trabalho apresenta um estudo sobre como é esse processo e como esse processo auxilia os problemas de Aprendizado de Máquina, além de mostrar resultados que incentivam o uso dessas técnicas. Finalmente, esse trabalho propõe a implementação de um serviço para selecionar características em meio a uma base utilizando modelos de seleção de características.

 \vspace{\onelineskip}
    
 \noindent
 \textbf{Palavras-chaves}: Seleção de características, Aprendizado de Máquina, Modelos de 
 Seleção de Características.
\end{resumo}
